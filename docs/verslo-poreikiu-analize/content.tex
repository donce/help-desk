\section{Verslo proceso analizė}

	\subsection{Išorinė verslo proceso analizė}

		\subsubsection{Proceso įeitys ir išeitys}

			Įeitis - krepinys - parametrai:
			\begin{itemize}
			\item Klientas
			\item Kreipinio gavimo būdas
			\item Tipas (INC ar REQ)
			\item Paslauga
			\item Kegistravimo laikas (data ir laikas)
			\item Išsprendimo (ar atsisakymo išspręsti) laikas
			\item Realiai sprendimui sugaištas laikas
			\item Jei reikia – nuoroda į prieš tai registruotą kreipinį
			\end{itemize}

			Išeitys, priklausomai nuo aplinkybių gali būti:
			\begin{itemize}
				\item Išspręsta problema (jei tai kreipinys - incidentas(INC))
				\item Suteikta informacija (jei tai kreipinys - prašymas(REQ))
				\item Atmestas kreipinys
			\end{itemize}

		\subsubsection{Grėsmės}
			\begin{itemize}
				\item Problema gali būti neišsprendžiama
				\item Problemos gali nepavykti išspręsti laiku
				\item Klientui gali nepavykti pateikti užklausos
			\end{itemize}
			
		\subsubsection{Efektyvumo matavimas}
		
			Išoriškai sistemos efektyvumas gali būti matuojamas pagal šitaiuos kriterijus:
			\begin{itemize}
				\item Užduoties pateikimo trukmė
					\subitem telefonu
					\subitem savitarnos svetainėje
				\item Užduoties įvykdymo trukmė
			\end{itemize}

		\subsubsection{Galimybės}

			\begin{itemize}
				\item Lengvas kreipinių pateikimas savitarnos svetainėje
					\subitem Supaprastinti užklausų pateikimo vartotojo sąsają
					\subitem Supaprastinti užklausų atsakymų gavimą
				\item Sutrumpintas laikas sugaištas užklausos vykdymui
			\end{itemize}

	\subsection{Vidinė verslo proceso analizė}

		\subsubsection{Agentai}
			Procese veikiantys agentai:
			\begin{itemize}
				\item Klientas
				\item Vadovas
				\item Administratorius
				\item Inžinierius
			\end{itemize}

		\subsubsection{Naudojimo atvejai}

			Sistemos naudojimo atvejai:
			\begin{itemize}
				\item Pateikti užklausą (REQ)
				\item Pateikti užklausą (INC)
				\item Pranešti apie incidentą
				\item Registruoti pateiktą užklausą telefonu/el. paštu
				\item Paskirti užklausą inžinieriui
				\item Išspręsti užklausą
				\item Atmesti užklausą
				\item Perskirti sprendžiamą užklausą iš bet kurio inžinieriaus bet kuriam kitam
				\item Tvarkyti Paslaugų, Klientų, Darbuotojų ir Sutarčių registrus
			\end{itemize}
			
			Skirtingi agentai gali atlikti skirtingus veiksmus.
			Tai iliustruoja naudojimo atvejų (use-case) diagrama pateikta \referToPicture{usecase.png}

			\insertPicture[0.5]{usecase.png}{Naudojimo atvejų diagrama}

		\subsubsection{Efektyvumo matavimas}
		
			Proceso efektyvumas matuojamas šiais kriterijais:
			\begin{itemize}
				\item Užklausos įvykdymo trukmė
				\item Įvykdytų užklausų kiekis
				\item Įvykdytų užklausų per tam tikrą laiko tarpą kiekis
				\item Laikas nuo užklausos pateikimo iki perdavimo inžinieriui
				\item Laikas nuo užklausos pateikimo iki vykdymo pradžios 
			\end{itemize}
			
		\subsubsection{Galimybės}

			Proceso efektyvumą galima pagerinti skiriant dėmesį šioms sritims:
			\begin{itemize}
				\item Užklausų paskirstymas inžinieriams pagal jų užimtumą
				(t.y. visada skirti užklausą mažiausiai užimtiems inžinieriams)
				\item Užklausų prioritetizavimas
				\item Užklausų sudėtingumo įvertinimas
				(t.y. negalima skirti vienam inžinieriui daug sudėtingų užduočių - geriau jas paskirstyti)
				\item Inžinieriaus darbo efektyvumo įvertinimas
				(spręsti kiek ir kokio sudėtingumo užduočių skirti pagal inžinieriaus efektyvumo įvertinimą)
			\end{itemize}
