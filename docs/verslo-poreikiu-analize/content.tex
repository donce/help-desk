\section{Verslo proceso analizė}

	\subsection{Išorinė verslo proceso analizė}

		\subsubsection{Proceso įeitys ir išeitys}

			Įeitis - krepinys - parametrai:
			\begin{itemize}
			\item Klientas
			\item Kreipinio gavimo būdas
			\item Tipas (INC ar REQ)
			\item Paslauga
			\item Kegistravimo laikas (data ir laikas)
			\item Išsprendimo (ar atsisakymo išspręsti) laikas
			\item Realiai sprendimui sugaištas laikas
			\item Jei reikia – nuoroda į prieš tai registruotą kreipinį
			\end{itemize}

			Išeitys, priklausomai nuo aplinkybių gali būti:
			\begin{itemize}
				\item Išspręsta problema (jei tai incidentas)
				\item Suteikta informacija (jei tai prašymas(REQ))
				\item Atmestas kreipinys
			\end{itemize}

		\subsubsection{Matavimai}

		\subsubsection{Grėsmės}
			\begin{itemize}
				\item Problema gali būti neišsprendžiama
				\item Problemos gali nepavykti išspręsti laiku
				\item Klientui gali nepavykti pateikti užklausos
			\end{itemize}

		\subsubsection{Galimybės}

			\begin{itemize}
				\item Palengvinti kreipinių pateikimą 
			\end{itemize}

	\subsection{Vidinė verslo proceso analizė}

		\subsubsection{Agentai}

			\begin{itemize}
				\item Klientas
				\item Vadovas
				\item Administratorius
				\item Inžinierius
			\end{itemize}

			\subsubsection{Naudojimo atvejai}

			Sistemos naudojimo atvejai:
			\begin{itemize}
				\item Pateikti užklausą (REQ)
				\item Pateikti užklausą (INC)
				\item Pranešti apie incidentą
				\item Registruoti pateiktą užklausą telefonu/el. paštu
				\item Paskirti užklausą inžinieriui
				\item Išspręsti užklausą
				\item Atmesti užklausą
				\item Perskirti sprendžiamą užklausą iš bet kurio inžinieriaus bet kuriam kitam
				\item Tvarkyti Paslaugų, Klientų, Darbuotojų ir Sutarčių registrus
			\end{itemize}



