%%
%% requirements.tex
%% Reikalavimų specifikacijos šablonas
%%
%% 'main.tex' turi būti pridėtas 'requirements' paketas ('requirements.sty') 

%% Naudojimas:
%%
%%	\subsection{Funkciniai programų sistemos reikalavimai}
%%		\requirements{F}{
%%			\req{tekstas1}
%%			\req{tekstas2}
%%		}

\section{Reikalavimai}

\subsection{Funkciniai programų sistemos reikalavimai}
	\requirements{F}{
		\req{Sistema turi suteikti galimybę atstovui pateikti kreipinį savitarnos svetainėje}
		\req{Sistema turi suteikti galimybę atstovui pateikti kreipinį telefonu (per administratorių)}
		\req{Sistema turi suteikti galimybę atstovui pateikti kreipinį el. paštu (per administratorių)}
		\req{Sistema turi suteikti galimybę administratoriui užregistruotį telefonu ar el. paštu pateiktą kreipinį}
		
	}	
	
\subsection{Nefunkciniai sistemos reikalavimai}

	\requirements{NF}{
		\req{Sistemoje turi būti apibrėžta "paslaugos" esybė}
		\req{Paslaugos esybei turi būti apibrėžti parametrai: 
			paslaugos kodas, aprašas, skirtingų kreipinių tipų išsprendimo terminai (valandomis)}
		\req{Paslaugos esybei turi būti galimybė pridėti papildomus parametrus}
		\req{Sistemoje turi būti apibrėžta "kliento" esybė}
		\req{Kliento esybei turi būti apibrėžti parametrai:
			kliento kodas, įmonės pavadinimas, adresas, atstovų el. pašto adresai, telefono numeriai}
	}

\subsection{Interfeiso reikalavimai}

\subsubsection{Dalykinės srities metaforų reikalavimai}

	\requirements{I-DSM}{
		\req{Juridinis asmuo užsisakęs paslaugą(-as) vadinamas "klientu"}
		\req{Kliento įgaliotas asmuo vadinamas "atstovu"}
		\req{Kliento pateiktas pranešimas apie gedimą arba klausimas vadinamas "kreipiniu"}
		\req{Kreipinys dėl paslaugos neveikimo vadinamas "incidentu" ("INC")}
		\req{Kreipinys dėl informacijos suteikimo vadinamas "paklausimu" ("REQ")}
	}