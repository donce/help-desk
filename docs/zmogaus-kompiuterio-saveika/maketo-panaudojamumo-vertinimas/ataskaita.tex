\section{Euristinio tikrinimo ataskaita}

\subsection{Santrauka}
	%% TODO: Santrauka
	
\subsection{Vertinimo aplinka}
	%% TODO: Aprašyti priemones naudotas vertinant (kompiuterius, programas - maketo peržiūros būdą)
	
\subsection{Vertinamos užduotys}
	%% TODO: Vertinamos užduotys
	
\subsection{Teigiami įspūdžiai}
	%% TODO: Teigiamos maketo savybės
	
\subsection{Pagrindinių problemų analizė}
	%% TODO: Pagrindinių problemų analizė
	
\subsection{Defektų sąrašas}
	%% TODO: Defektų sąrašas
	
	\begin{table}[ht] 
	\caption{Nonlinear Model Results} % title of Table 
	\centering % used for centering table 
	\begin{tabular}{c c c c} % centered columns (4 columns) 
	\hline\hline %inserts double horizontal lines 
	Nr. & Problema & Rado & Principas \\ [0.5ex] % inserts table 
	%heading 
	\hline % inserts single horizontal line 
	1 & Administratorius>Registrai, mygtukas "Importuoti" per arti kitų mygtukų. & U, K & Klaidų vengimas \\ % inserting body of the table 
	2 & Kortelės "Pagrindinis" turinys yra be reikalo pateikiamas atidarius puslapį. & U, K & Naudojimo lankstumas ir efektyvumas \\
	3 & Administratorius neturi galimybės peržiūrėti priskirtų kreipinių. & U, K & Sistemos atitikimas realiai situacijai \\
	4 & Nėra būdo grįžti iš kreipinio priskirimo vaizdo & U, K & Laisvas vartotojo valdymo dialogas. \\
	5 & Kreipinių sąrašuose yra neįvardintos paslaugos & U, K & Sistemos atitikimas realiai situacijai \\
	6 & Administratoriui pasirinkus klientą nėra galimybės nurodyti paslaugą, kuriai skirta užklausa. & U, K & Sistemos atitikimas realiai situacijai \\
	7 & Nėra galymybės greitai rasti visus neišspręstus kreipinius & U, K & Sistemos neatitikimas realiai situacijai. \\
	8 & Nereikalingas žodis "kreipiniai" - jis akivaizdus. & U, K & Estetiškas dizainas. \\
	9 & Klientas, mygtukas "Naujas kreipinys" yra netinkamoje vietoje. & U, K & Darnus dialogas \\
	10 & Statistikos peržiūroje trūksta metų pasirinkimo galymybių. & U, K & Sistemos atitikimas realiai situacijai \\
	\hline %inserts single line 
	\end{tabular} 
	\label{table:nonlin} % is used to refer this table in the text 
	\end{table} 

	