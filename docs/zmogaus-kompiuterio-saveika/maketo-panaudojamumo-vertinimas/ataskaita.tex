\section{Euristinio tikrinimo ataskaita}

\subsection{Santrauka}
	%% TODO: Santrauka
	
\subsection{Vertinimo aplinka}
	%% TODO: Aprašyti priemones naudotas vertinant (kompiuterius, programas - maketo peržiūros būdą)
	\begin{table}[ht] 
	\caption{Nonlinear Model Results} % title of Table 
	\centering % used for centering table 
	\begin{tabular}{c c c c c c} % centered columns (6 columns) 
	\hline\hline %inserts double horizontal lines 
	Vertintojas & U & K & D & R & O \\ [0.5ex] % inserts table 
	%heading 
	\hline % inserts single horizontal line 
	Amžius & 21 & 21\\ % inserting body of the table 
	Lytis & M & V\\
	OS & Windows 7 & Ubuntu 13.10\\
	Ekrano spalvos & 32b & 32b\\
	Skiriamoji geba & 1336x768 & 1336x768\\
	Ekrano dydis & 15.6" & 15.6"\\
	Vertinimo data & 2013-11-25 & 2013-11-25\\
	Vertinimo laikas & 18:00- & 18:00-\\
	\hline %inserts single line 
	\end{tabular} 
	\label{table:nonlin} % is used to refer this table in the text 
	\end{table} 
	
\subsection{Vertinamos užduotys}
	%% TODO: Vertinamos užduotys
	\begin{enumerate}
	\item Prisijungimas
	\item Kreipinio išsprendimas
	\item Nusprendimas neišspręsti kreipinio
	\item Kreipinio grąžinimas
	\item Pasiekti inžinieriaus veiksmus
	\item Pasiekti administratoriaus veiksmus
	\item Perskirti kreipinį kitam inžinieriui
	\item Peržiūrėti statistiką
	\item Kreipinio registravimas
	\item Kreipinio priskyrimas inžinieriui
	\item Registrų tvarkymas
	\item Duomenų importavimas iš struktūros
	\item Pateikti kreipinį savitarnos svetainėje
	\item Įvertinti atliktą kreipinį
	\end{enumerate}
	
\subsection{Teigiami įspūdžiai}
	%% TODO: Teigiamos maketo savybės
	\begin{enumerate}
	\item Kiekviename interfeise yra sistemos meniu juosta, leidžianti lengviau susiorientuoti sistemoje bei greitai pasiekti sistemos dalis.
	Šis paprastas komponentas yra kiekviename lange, todėl tenkina darnaus dialogo euristiką.
	Funkcijų grupavimas į apibendrinančias grupes atitinka atpažinimo vietoje įsiminimo euristiką.
	\item Registrai yra suskirstyti į korteles pagal esybes, o tai leidžia juos visus patogiai pateikti viename puslapyje ir padeda lengviau susiorientuoti tarp jų.
	Šios kortelės atitinka atpažinimo vietoje įsiminimo euristiką.
	\item Inžinieriui kreipinio informacija yra pateikiama labai patogiai. Pagrindiniai kreipinio informacijos elementai yra atskiriami nuo viso kreipinio aprašymo.
	Tai padeda lengviau atkreipti dėmesį į tam momentui reikalingą informaciją.
	Toks paskirstymas atitinka darnaus dialogo euristiką.
	\item Kliento interfeise kreipinių peržiūra yra labai patogi.
	Tame pačiame lange galima matyti viską: kreipinių sąrašą, kreipinio aprašymą bei atsakymą į kreipinį.
	Šis išdėstymas nereikalauja sudėtingos navigacijos tarp sistemos puslapių.
	Toks vaizdavimo būdas išpildo sistemos atitikimo realiai situacijai euristiką.
	\end{enumerate}
	
\subsection{Pagrindinių problemų analizė}
	%% TODO: Pagrindinių problemų analizė
	
\subsection{Defektų sąrašas}
	%% TODO: Defektų sąrašas
	
	\begin{table}[ht] 
	\caption{Nonlinear Model Results} % title of Table 
	\centering % used for centering table 
	\begin{tabular}{c c c c} % centered columns (4 columns) 
	\hline\hline %inserts double horizontal lines 
	Nr. & Problema & Rado & Principas \\ [0.5ex] % inserts table 
	%heading 
	\hline % inserts single horizontal line 
	1 & Administratorius>Registrai, mygtukas "Importuoti" per arti kitų mygtukų. & U, K & Klaidų vengimas \\ % inserting body of the table 
	2 & Kortelės "Pagrindinis" turinys yra be reikalo pateikiamas atidarius puslapį. & U, K & Naudojimo lankstumas ir efektyvumas \\
	3 & Administratorius neturi galimybės peržiūrėti priskirtų kreipinių. & U, K & Sistemos atitikimas realiai situacijai \\
	4 & Nėra būdo grįžti iš kreipinio priskirimo vaizdo & U, K & Laisvas vartotojo valdymo dialogas. \\
	5 & Kreipinių sąrašuose yra neįvardintos paslaugos & U, K & Sistemos atitikimas realiai situacijai \\
	6 & Administratoriui pasirinkus klientą nėra galimybės nurodyti paslaugą, kuriai skirta užklausa. & U, K & Sistemos atitikimas realiai situacijai \\
	7 & Nėra galymybės greitai rasti visus neišspręstus kreipinius & U, K & Sistemos neatitikimas realiai situacijai. \\
	8 & Nereikalingas žodis "kreipiniai" - jis akivaizdus. & U, K & Estetiškas dizainas. \\
	9 & Klientas, mygtukas "Naujas kreipinys" yra netinkamoje vietoje. & U, K & Darnus dialogas \\
	10 & Statistikos peržiūroje trūksta metų pasirinkimo galymybių. & U, K & Sistemos atitikimas realiai situacijai \\
	11 & Netikslinga laikyti sistemos duomenų ištrynimo pasirinkimą duomenų importavimo lange. & U, K & Sistemos atitikimas realiai situacijai \\
	\hline %inserts single line 
	\end{tabular} 
	\label{table:nonlin} % is used to refer this table in the text 
	\end{table} 