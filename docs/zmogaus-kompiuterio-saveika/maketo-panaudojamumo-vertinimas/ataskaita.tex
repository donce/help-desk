\section{Euristinio tikrinimo ataskaita}

\subsection{Santrauka}

	\subsubsection{Vertinimo vykdytojai}

	Vertinimą atliko panaudojamumo ekspertų komanda:
	%% TODO: list
		Ugnė Laima Čižiūtė
		Karolis Jocevičius
		Donatas Kučinskas
		Rytis Karpuška
		Oleg Koldun
	
	Visų vertinimo vykdytojų amžius buvo labai panašus.
	Visi vykdytojai yra studentai, studijuojantys Programų Sistemų 3 kursą Vilniaus Universitete.
	Keturi vykdymo nariai yra vyriškos lyties, vienas - moteriškos.

	\subsubsection{Vertinimo aplinka}

	Vertinimo komanda naudojo pakankamai skirtingas vertinimo aplinkas - vertinimui buvo naudotos tiek Windows, tiek Linux operacinės sistemos.
	Populiariausia operacinė sistema - Ubuntu (skaičiuojant su tokiomis atšakomis kaip Xubuntu).
	Vertinimui naudojami monitoriai buvo skirtingų dydžių bei rezoliucijų, tačiau visi turėjo 32 bitus spalvų.

	\subsubsection{Vertinimo eiga}

	Prieš maketo testavimą, buvo sudarytas užduočių sąrašas.
	Kiekvienas vertinimo narys kiekvieną užduotį turėjo perskaityti ir ją vaizduotėje pagal maketą atlikti.
	Atliekant veiksmus, reikėjo išskirti labiausiai patikusius maketo dalykus, juos pasižymėti.
	Rastus maketo nepatogumus ir problemas taipogi pasižymėti ir surašyti lentelėje, įvertinti jų svarbą.

	\subsubsection{Vertinimo rezultatai}

	Atlikus vertinimą, buvo rasta tiek teigiamų, tiek ir neigiamų savybių.
	%TODO: vertinimo rezultatai
	
		\begin{table}[ht] 
		\caption{Nonlinear Model Results} % title of Table 
		\centering % used for centering table 
			\begin{tabular}{c c} % centered columns (2 columns) 
		\hline\hline %inserts double horizontal lines 
		Kodas & Vertintojas\\ [0.5ex] % inserts table 
		%heading 
		\hline % inserts single horizontal line 
		U & Ugnė Laima Čižiūtė
		K & Karolis Jocevičius
		D & Donatas Kučinskas
		R & Rytis Karpuška
		O & Oleg Koldun
		\hline %inserts single line 
		\end{tabular} 
		\label{table:nonlin} % is used to refer this table in the text 
		\end{table} 
		
	\subsubsection{Problemos ir taisymo rekomendacijos}
	
		\begin{table}[ht] 
		\caption{Nonlinear Model Results} % title of Table 
		\centering % used for centering table 
			\begin{tabular}{c c c} % centered columns (3 columns) 
		\hline\hline %inserts double horizontal lines 
		Nr. & Problema & Sprendimas\\ [0.5ex] % inserts table 
		%heading 
		\hline % inserts single horizontal line 
		1 & Administratorius>Registrai, mygtukas "Importuoti" per arti kitų mygtukų. & Jį reikėtų perkelti į dešinę pusę.\\ %%+
		2 & Kortelės "Pagrindinis" turinys yra be reikalo pateikiamas atidarius puslapį. & Šios kortelės turinys turėtų būti pateikiamas paspaudus ant vartotojo vardo dešiniajame kampe.\\%%+
		3 & Administratorius neturi galimybės peržiūrėti priskirtų kreipinių. & Suteikti galimybę pasirinkti tarp priskirtų ir nepriskirtų kreipinių rodymo.\\ %%+
		4 & Nėra būdo grįžti iš kreipinio priskyrimo lango bei struktūros importavimo lango. & Kairiajame kampe pridėti mygtuką "Grįžti".\\ %%+
		5 & Kreipinių sąrašuose yra neįvardintos paslaugos & Pridėti stulpelį su paslaugos pavadinimu.\\%%+
		6 & Administratoriui kuriant kreipinį nėra galimybės nurodyti paslaugą, kuriai skirtas kreipinys. & Reikia pridėti išskleidžiamą paslaugų sąrašą.\\%%+
		7 & Inžinierius, nėra galimybės greitai rasti visus neišspręstus kreipinius & Įdiegti sistemoje rikiavimą.\\ %%+
		8 & Klientas, nereikalingas žodis "kreipiniai" - jis akivaizdus. & Pašalinti žodį "kreipiniai".\\%%?
		9 & Klientas, mygtukas "Naujas kreipinys" yra netinkamoje vietoje. & Jį reikėtų perkelti į dešinę pusę.\\%%?
		10 & Statistikos peržiūroje trūksta metų pasirinkimo galimybės. & Reikia pridėti išskleidžiamą metų pasirinkimo sąrašą.\\ %%+
		11 & Netikslinga laikyti sistemos duomenų ištrynimo pasirinkimą duomenų importavimo lange. & Duomenų ištrynimo mygtuką sukurti sistemos administravimo lange.\\%%+
		\hline %inserts single line 
		\end{tabular} 
		\label{table:nonlin} % is used to refer this table in the text 
		\end{table} 
	
\subsection{Vertinimo aplinka}
	%% TODO: Aprašyti priemones naudotas vertinant (kompiuterius, programas - maketo peržiūros būdą)
	\begin{table}[ht] 
	\caption{Nonlinear Model Results} % title of Table 
	\centering % used for centering table 
	\begin{tabular}{c c c c c c} % centered columns (6 columns) 
	\hline\hline %inserts double horizontal lines 
	Vertintojas & U & K & D & R & O \\ [0.5ex] % inserts table 
	%heading 
	\hline % inserts single horizontal line 
	Amžius & 21 & 21 & 21\\ % inserting body of the table 
	Lytis & M & V & V\\
	OS & Windows 7 & Ubuntu 13.10 & Xubuntu 13.10\\
	Ekrano spalvos & 32b & 32b & 32b\\
	Skiriamoji geba & 1336x768 & 1336x768 & 1600x900\\
	Ekrano dydis & 15.6" & 15.6" & 14"\\
	Vertinimo data & 2013-11-25 & 2013-11-25\\
	Vertinimo laikas & 18:00- & 18:00-\\
	\hline %inserts single line 
	\end{tabular} 
	\label{table:nonlin} % is used to refer this table in the text 
	\end{table} 
	
\subsection{Vertinamos užduotys}
	%% TODO: Vertinamos užduotys
	\begin{enumerate}
	\item Prisijungimas
	\item Kreipinio išsprendimas
	\item Nusprendimas neišspręsti kreipinio
	\item Kreipinio grąžinimas
	\item Pasiekti inžinieriaus veiksmus
	\item Pasiekti administratoriaus veiksmus
	\item Perskirti kreipinį kitam inžinieriui
	\item Peržiūrėti statistiką
	\item Kreipinio registravimas
	\item Kreipinio priskyrimas inžinieriui
	\item Registrų tvarkymas
	\item Duomenų importavimas iš struktūros
	\item Pateikti kreipinį savitarnos svetainėje
	\item Įvertinti atliktą kreipinį
	\end{enumerate}
	
\subsection{Teigiami įspūdžiai}
	%% TODO: Teigiamos maketo savybės
	\begin{enumerate}
	\item Kiekviename interfeise yra sistemos meniu juosta, leidžianti lengviau susiorientuoti sistemoje bei greitai pasiekti sistemos dalis.
	Šis paprastas komponentas yra kiekviename lange, todėl tenkina darnaus dialogo euristiką.
	Funkcijų grupavimas į apibendrinančias grupes atitinka atpažinimo vietoje įsiminimo euristiką.
	\item Registrai yra suskirstyti į korteles pagal esybes, o tai leidžia juos visus patogiai pateikti viename puslapyje ir padeda lengviau susiorientuoti tarp jų.
	Šios kortelės atitinka atpažinimo vietoje įsiminimo euristiką.
	\item Inžinieriui kreipinio informacija yra pateikiama labai patogiai. Pagrindiniai kreipinio informacijos elementai yra atskiriami nuo viso kreipinio aprašymo.
	Tai padeda lengviau atkreipti dėmesį į tam momentui reikalingą informaciją.
	Toks paskirstymas atitinka darnaus dialogo euristiką.
	\item Kliento interfeise kreipinių peržiūra yra labai patogi.
	Tame pačiame lange galima matyti viską: kreipinių sąrašą, kreipinio aprašymą bei atsakymą į kreipinį.
	Šis išdėstymas nereikalauja sudėtingos navigacijos tarp sistemos puslapių.
	Toks vaizdavimo būdas išpildo sistemos atitikimo realiai situacijai euristiką.
	\end{enumerate}
	
\subsection{Pagrindinių problemų analizė}
	%% TODO: Pagrindinių problemų analizė
	
	\subsubsection{Administratorius neturi galimybės peržiūrėti priskirtų kreipinių}
		
		Administratoriui yra pateiktas naujų kreipinių sąrašas.
		Tačiau, peržiūrėti priskirtus kreipinius inžinieriams galimybės nėra.
		Čia pažeista sistemos atitikimo realiai situacijai euristika.
		Šia problemą galima išspręsti pridedant du mygtukus pasirinkimui tarp priskirtų ir nepriskirtų kreipinių.
		
	\subsubsection{Nėra būdo grįžti iš kreipinio priskyrimo lango bei struktūros importavimo lango}
	
		Administratoriaus interfeiso kreipinio priskyrimo bei struktūros importavimo languose nėra grįžimo arba atšaukimo mygtukų.
		Todėl nusprendus nevykdyti pasirinktos funkcijos nėra galimybės grįžti į ankstesnį vaizdą.
		Čia pažeista laisvo vartotojo valdymo dialogo euristika.
		Šią problemą galima išspręsti pridedant langų kairiuose kampuose mygtukus "Grįžti".
	
\subsection{Defektų sąrašas}
	%% TODO: Defektų sąrašas
	
	\begin{table}[ht] 
	\caption{Nonlinear Model Results} % title of Table 
	\centering % used for centering table 
	\begin{tabular}{c c c c c} % centered columns (5 columns) 
	\hline\hline %inserts double horizontal lines 
	Nr. & Problema & Rado & Principas & Prioritetas\\ [0.5ex] % inserts table 
	%heading 
	\hline % inserts single horizontal line 
	1 & Administratorius>Registrai, mygtukas "Importuoti" per arti kitų mygtukų. & U, K & Klaidų vengimas & Vidutinis\\%%+
	2 & Kortelės "Pagrindinis" turinys yra be reikalo pateikiamas atidarius puslapį. & U, K & Naudojimo lankstumas ir efektyvumas & Vidutinis\\%%+
	3 & Administratorius neturi galimybės peržiūrėti priskirtų kreipinių. & U, K & Sistemos atitikimas realiai situacijai & Aukštas\\%%+
	4 & Nėra būdo grįžti iš kreipinio priskyrimo lango bei struktūros importavimo lango & U, K & Laisvas vartotojo valdymo dialogas & Aukštas\\%%+
	5 & Kreipinių sąrašuose yra neįvardintos paslaugos & U, K & Sistemos atitikimas realiai situacijai & Vidutinis\\%%+
	6 & Administratoriui kuriant kreipinį nėra galimybės nurodyti paslaugą, kuriai skirtas kreipinys. & U, K & Sistemos atitikimas realiai situacijai & Vidutinis\\%%+
	7 & Nėra galimybės greitai rasti visus neišspręstus kreipinius & U, K & Sistemos neatitikimas realiai situacijai & Žemas\\%%+
	8 & Nereikalingas žodis "kreipiniai" - jis akivaizdus. & U, K & Estetiškas dizainas & Žemas\\%%?
	9 & Klientas, mygtukas "Naujas kreipinys" yra netinkamoje vietoje. & U, K & Darnus dialogas & Žemas\\%%?
	10 & Statistikos peržiūroje trūksta metų pasirinkimo galimybės. & U, K & Sistemos atitikimas realiai situacijai & Vidutinis\\%%+
	11 & Netikslinga laikyti sistemos duomenų ištrynimo pasirinkimą duomenų importavimo lange. & U, K & Sistemos atitikimas realiai situacijai & Žemas\\%%+
	12 & Nėra būdo prisijungti pamiršus slaptažodį. & D & & Vidutinis\\%%+
	13 & Nėra būdo prirašyti komentarą išsprendžiant/atmetant kreipinį. & D & & Vidutinis\\%%+
	14 & Negalima peržiūrėti statistikos kitokiam laikotarpiui nei vienam mėnesiui. & D & & Vidutinis\\%%+
	15 & Registruojant kreipinį, pasirinkti klientą iš sąrašo pagal vardą nepatogu. & D & & Vidutinis\\%%+
	16 & Priskiriant kreipinį, pasirinkti darbuotoją iš sąrašo pagal vardą nepatogu. & D & & Vidutinis\\%%+
	17 & Nėra būdo sukurti naują registrą. & D & & Vidutinis\\%%+
	18 & Nėra būdo redaguoti registrus. & D & & Vidutinis\\%%+
	19 & Nėra būdo ištrinti registrus. & D & & Vidutinis\\%%+
	20 & Nėra būdo filtruoti registrus pagal jų laukus. & D & & Vidutinis\\%%+
	21 & Klientui nėra būdo filtruoti kreipinius pagal jų išsprendimą. & D & & Vidutinis\\%%+
	\hline %inserts single line 
	\end{tabular} 
	\label{table:nonlin} % is used to refer this table in the text 
	\end{table} 
