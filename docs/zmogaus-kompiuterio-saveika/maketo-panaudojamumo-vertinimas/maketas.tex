
\section{Patobulintas prototipas}
	%% TODO: Patobulintas prototipas

\subsection{Inžinierius}
	
	\insertPicture[0.8]{MaketasInzinieriusPagrindinis.png}{Inžinieriaus pradinio vaizdo maketas}
	
	\insertPicture[0.8]{MaketasInzinieriusKreipiniuSarasas.png}{Inžinieriui pateikiamas kreipinių sąrašas}
	
	\insertPicture[0.8]{MaketasInzinieriusKreipinioRodymas.png}{Inžinieriui pateikiamas detalesnis kreipinio aprašymas}

	\subsubsection{Keipinio Išsprendimas}
	
	Inžinierius norėdamas pažymėti kreipinį kaip išspręstą turi atlikti šiuos veikmus:
	\begin{itemize}
		\item Paspausti ant kortelės "Spręsti kreipinius" \referToPicture{MaketasInzinieriusPagrindinis.png}
		\item Iš pateikto kreipinių sąrašo išsirinkti sau dominantį kreipinį \referToPicture{MaketasInzinieriusKreipiniuSarasas.png}
		\item Paspausti mygtuką "Pažymėti išspręstu" \referToPicture{MaketasInzinieriusKreipinioRodymas.png}
	\end{itemize}
	
	\subsubsection{Nusprendimas neišspręsti kreipinio}
	
	Inžinierius norėdamas nuspręsti neišspręsti kreipinio turi atlikti šiuos veiksmus:
	
	\begin{itemize}
		\item Paspausti ant kortelės "Spręsti kreipinius" \referToPicture{MaketasInzinieriusPagrindinis.png}
		\item Iš pateikto kreipinių sąrašo išsirinkti sau dominantį kreipinį \referToPicture{MaketasInzinieriusKreipiniuSarasas.png}
		\item Paspausti mygtuką "Atmesti Kreipinį" \referToPicture{MaketasInzinieriusKreipinioRodymas.png}
	\end{itemize}
	
	\subsubsection{Kreipinio grąžinimas}
	
	Inžinierius norėdamas gražinti kreipinį administratoriui turi atlikti šiuos veikmus:
	
	\begin{itemize}
		\item Paspausti ant kortelės "Spręsti kreipinius" \referToPicture{MaketasInzinieriusPagrindinis.png}
		\item Iš pateikto kreipinių sąrašo išsirinkti sau dominantį kreipinį \referToPicture{MaketasInzinieriusKreipiniuSarasas.png}
		\item Paspausti mygtuką "Perduoti Administratoriui" \referToPicture{MaketasInzinieriusKreipinioRodymas.png}
	\end{itemize}

\subsection{Vadovas}

	\subsubsection{Pasiekti administratoriaus, inžinieriaus veiksmus}
		
	\insertPicture[0.8]{maketas-vadovas-startas.png}{Vadovo pradinis vaizdas}
	
	Vadovui pasiekti administratoriaus bei inžinieriaus veiksmus galima pasinaudojus viršuje esančia juosta su mygtukais kiekvienam iš vaizdų.
	Pele paspaudus mygtuką "Administravimas" pereinama į Administratoriaus vaizdą (žr. Administratoriaus aprašymą).
	Paspaudus mygtuką "Kreipiniai" pereinama į Inžinieriaus vaizdą (žr. Inžinieriaus aprašymą).
	Pradinio Vadovo interfeiso maketas pateiktas \referToPicture{maketas-vadovas-startas.png}
	
	\subsubsection{Peržiūrėti statistiką}	
	
	\insertPicture[0.8]{maketas-vadovas-apzvalga}{Vadovo "Apžvalgos" vaizdas} 	
	
	\insertPicture[0.8]{maketas-statistika}{Statistikos vaizdas} 
	
	Vadovui pasiekti vėluojančių kreipinių statistiką galima iš "Apžvalgos" ekrano (Žr. \referToPicture{maketas-vadovas-apzvalga}), paspaudus statistikos mygtuką.
	Statistika pateikiama diagramomis bei sąrašais. Žr. \referToPicture{maketas-statistika}
	
	\textit{Pastaba: "Apžvalga" - vadovo pradinis ekranas - rezervuotas papildomam funkcionalumui, pagal kurį dokumentas bus papildytas.}
	
\subsection{Klientas}

	\subsubsection{Pateikti kreipinį}
	
	\insertPicture[0.8]{maketas-klientas-naujas_kreipinys.png}{Naujo kreipinio pateikimo maketas}
	
	Klientui, norint pateikti kreipinį, reikia paspausti mygtuką "Kreipiniai" ir tada mygtuką "Naujas kreipinys".
	Dešinėje lango pusėje atsiranda naujo kreipinio forma. Joje reikia užpildyti laukus:
	
	\begin{itemize}
		\item Įrašyti temą
		\item Aprašyti kreipinį
		\item Pasirinkti paslaugą
		\item Pasirinkti kreipinio tipą
	\end{itemize}
	
	Užpildžius naujo kreipinio formą reikia paspausti mygtuką "Siųsti".

	\subsubsection{Įvertinti atliktą kreipinį}
	
	\insertPicture[0.8]{maketas-klientas-ivertinti.png}{Atlikto kreipinio įvertinimas}
	
	Klientui, norint įvertinti atliktą kreipinį, reikia paspausti mygtuką "Kreipiniai" ir tada pasirinkti kreipinį.
	Dešinėje lango pusėje atsiranda laukas su išsiųstu kreipinio aprašymu, gautu atsakymu ir vertinimo forma.
	Vertinant atliktą kreipinį reikia perskaityti gautą atsakymą, vertinimo formoje pasirinkti įvertinimą ir paspausti mygtuką "Pateikti".
	
\subsection{Administratorius}

	\insertPicture[0.8]{Maketas_Administratorius_Pagrindinis.png}{Administratoriaus pradinio vaizdo maketas}

	\subsubsection{Kreipinio registravimas}

	\insertPicture[0.8]{Maketas_Administratorius_TvarkytiKreipinius.png}{Kreipinių tvarkymo maketas}
	\insertPicture[0.8]{Maketas_Administratorius_TvarkytiKreipinius_Registruoti.png}{Kreipinio registravimo maketas}

	Norėdamas užregistruoti kreipinį, administratorius turi atlikti šiuos veiksmus:
	\begin{itemize}
		\item Paspausti ant kortelės "Tvarkyti kreipinius" \referToPicture{Maketas_Administratorius_Pagrindinis.png}
		\item Paspausti ant mygtuko "Sukurti" \referToPicture{Maketas_Administratorius_TvarkytiKreipinius.png}
		\item Pasirinkti kreipinio tipą, klientą, gavimo būdą. Tada kreipinio pranešimą reikia įvesti į pateiktą teksto lauką ir paspausti mygtuką "Registruoti" \referToPicture{Maketas_Administratorius_TvarkytiKreipinius_Registruoti.png}
	\end{itemize}

	\subsubsection{Kreipinio paskyrimas inžinieriui}

	\insertPicture[0.8]{Maketas_Administratorius_TvarkytiKreipinius_Priskirti.png}{Kreipinio priskyrimo maketas}

	Norėdamas kreipinį paskirti inžinieriui, administratorius turi atlikti šiuos veiksmus:
	\begin{itemize}
		\item Paspausti ant kortelės "Tvarkyti kreipinius" \referToPicture{Maketas_Administratorius_Pagrindinis.png}
		\item Iš pateikto kreipinių sąrašo paspausti ant norimo kreipinio \referToPicture{Maketas_Administratorius_TvarkytiKreipinius.png}
		\item Iš pateikto darbuotojų sąrašo pasirinkti darbuotoją ir paspausti mygtuką "Priskirti" \referToPicture{Maketas_Administratorius_TvarkytiKreipinius_Priskirti.png}
	\end{itemize}

	\subsubsection{Duomenų importavimas iš struktūros}

	\insertPicture[0.8]{Maketas_Administratorius_Administruoti.png}{Administravimo lango maketas}
	\insertPicture[0.8]{Maketas_Administratorius_Administruoti_Importuoti.png}{Struktūros importavimo maketas}

	Norėdamas importuoti duomenis iš struktūros, administratorius turi atlikti šiuos veiksmus:
	\begin{itemize}
		\item Paspausti ant kortelės "Administravimas" \referToPicture{Maketas_Administratorius_Pagrindinis.png}
		\item Paspausti ant mygtuko "Importuoti struktūrą" \referToPicture{Maketas_Administratorius_Administruoti.png}
		\item Nurodyti struktūros failą, pasirinkti, ar ištrinti visus duomenis prieš įkeliant struktūrą, ir paspausti mygtuką "Importuoti" \referToPicture{Maketas_Administratorius_Administruoti_Importuoti.png}
	\end{itemize}

	\subsubsection{Registru tvarkymas}

	Administratoriui norint tvarkyti registrus reikia pereiti į registru langą. Pasirinkti reikiamą registrų tipą. 
	Priklausomai nuo reikiamo veiksmo:
	
	\begin{itemize}
		\item Šalinti irašą.
		\item Pridėti irašą.
		\item Redaguoti irašą.
		\item Duomenu importas.
	\end{itemize}
	
	\insertPicture[0.8]{admin-tvarkyti_registrus.png}{Registrų tvarkymo maketas}
	
	Jei veiksmas: šalinti irašą. Lange \referToPicture{admin-tvarkyti_registrus.png} reikia pasirinkti registro tipą ir ištrinti eilutę iš lentelės.

	Jei veiksmas: pridėti irašą. Lange \referToPicture{admin-tvarkyti_registrus.png} reikia pasirinkti registro tipą ir sukurti eilutę lentelėje.

	Jei veiksmas: redaguoti irašą. Lange \referToPicture{admin-tvarkyti_registrus.png} reikia pasirinkti registro tipą, pakeisti reikiamo registro eilutės duomenis.

	Jei veiksmas: duomenu importas. Lange \referToPicture{admin-tvarkyti_registrus.png} reikia spausti migtuka importuoti ir pasirinkti duomenų failą.

