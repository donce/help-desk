
\section{Maketas}

\subsection{Vadovas}

	\subsubsection{Pasiekti administratoriaus, inžinieriaus veiksmus}
		
	Vadovui pasiekti administratoriaus bei inžinieriaus veiksmus galima pasinaudojus viršuje esančia juosta su mygtukais kiekvienam iš vaizdų.
	Pele paspaudus mygtuką "Administravimas" pereinama į Administratoriaus vaizdą (žr. Administratoriaus aprašymą).
	Paspaudus mygtuką "Kreipiniai" pereinama į Inžinieriaus vaizdą (žr. Inžinieriaus aprašymą).
	Pradinio Vadovo interfeiso maketas pateiktas \referToPicture{maketas-vadovas-startas.png}
	
	\textit{Pastaba: "Apžvalga" - pradinis ekranas - rezervuotas papildomam funkcionalumui, pagal kurį dokumentas bus papildytas.}
	
	\insertPicture[0.8]{maketas-vadovas-startas.png}{Vadovo pradinis vaizdas}
	
\subsection{Klientas}

	\subsubsection{Pateikti kreipinį}
	
	Klientui norint pateikti kreipinį reikia pasinkti kreipinio tipą:
	
	\begin{itemize}
		\item paslaugos neveikimas (incidentas - INC)
		\item informacijos suteikimas (paklausimas - REQ)
	\end{itemize}
	
	Pasirinkus INC tipą pereinama į incidento aprašymo langą \referToPicture{maketas-klientas-INC.png}
	Kairėje lango pusėje reikia pasirinkti paslaugą, su kuria yra susijęs incidentas, o dešinėje jį aprašyti.
	Paspaudus mygtuką "Siųsti" kreipinys bus išsiųstas.
	Taip pat lango dešinėje apatinėje dalyje galima peržiūrėti išsiųstus INC tipo kreipinius, bei gautus atsakymus paspaudus ant jų.
	
	\insertPicture[0.8]{maketas-klientas-INC.png}
	
	Pasirinkus REQ tipą pereinama į paklausimo aprašymo langą \referToPicture{maketas-klientas-REQ.png}
	Viduryje lango yra didelis tekstinis laukas į kurį iš karto galimą rašyti norimą klausimą, o jį parašius išsiųsti, paspaudus mygtuką "Siųsti".
	Apatinėje lango dalyje galima peržiūrėti išsiųstus REQ tipo kreipinius, bei gautus atsakymus paspaudus ant jų.
	
	\insertPicture[0.8]{maketas-klientas-REQ.png}

\subsection{Administratorius}

	\subsubsection{Registru tvarkymas}
	
	Administratoriui norint tvarkyti registrus reikia pereiti į registru langą. Pasirinkti reikiamą registrų tipą. 
	Priklausomai nuo reikiamo veiksmo:
	
	\begin{itemize}
		\item Šalinti irašą.
		\item Pridėti irašą.
		\item Redaguoti irašą.
		\item Importuoti irašus.
	\end{itemize}
	
	\insertPicture[0.8]{adminas-registru_tvarkynmas_UI.png}
	
	Jei veiksmas: šalinti irašą, tai reikia pasirinkti irašą ir spausti migtuką "šalinti".

	Jei veiksmas: pridėti irašą, spausti "pridėti" ir užpildyti žemiau esančia lentelę reikiamo registro tipo duomenimis.

	Jei veiksmas: pridėti irašą, tai reikia pasirinkti irašą spausti "redaguoti" ir keisti žemiau esančios lentelės langelius reikiamo.
	
	Jei veiksmas: šalinti irašą, spausti "importuoti" ir standartiniame lange pasirinkti failą iš kurio importuoti.