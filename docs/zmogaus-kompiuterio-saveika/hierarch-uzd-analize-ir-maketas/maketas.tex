
\section{Maketas}

\subsection{Inžinierius}
	
	\insertPicture[0.8]{MaketasInzinieriusPagrindinis.png}{Inžinieriaus pradinio vaizdo maketas}
	
	\insertPicture[0.8]{MaketasInzinieriusKreipiniuSarasas.png}{Inžinieriui pateikiamas kreipinių sąrašas}
	
	\insertPicture[0.8]{MaketasInzinieriusKreipinioRodymas.png}{Inžinieriui pateikiamas detalesnis kreipinio aprašymas}

	\subsubsection{Keipinio Išsprendimas}
	
	Inžinierius norėdamas pažymėti kreipinį kaip išspręstą turi atlikti šiuos veikmus:
	\begin{itemize}
		\item Paspausti ant kortelės "Spręsti kreipinius" \referToPicture{MaketasInzinieriusPagrindinis.png}
		\item Iš pateikto kreipinių sąrašo išsirinkti sau dominantį kreipinį \referToPicture{MaketasInzinieriusKreipiniuSarasas.png}
		\item Paspausti mygtuką "Pažymėti išspręstu" \referToPicture{MaketasInzinieriusKreipinioRodymas.png}
	\end{itemize}
	
	\subsubsection{Nusprendimas neišspręsti kreipinio}
	
	Inžinierius norėdamas nuspręsti neišspręsti kreipinio turi atlikti šiuos veiksmus:
	
	\begin{itemize}
		\item Paspausti ant kortelės "Spręsti kreipinius" \referToPicture{MaketasInzinieriusPagrindinis.png}
		\item Iš pateikto kreipinių sąrašo išsirinkti sau dominantį kreipinį \referToPicture{MaketasInzinieriusKreipiniuSarasas.png}
		\item Paspausti mygtuką "Atmesti Kreipinį" \referToPicture{MaketasInzinieriusKreipinioRodymas.png}
	\end{itemize}
	
	\subsubsection{Kreipinio grąžinimas}
	
	Inžinierius norėdamas gražinti kreipinį administratoriui turi atlikti šiuos veikmus:
	
	\begin{itemize}
		\item Paspausti ant kortelės "Spręsti kreipinius" \referToPicture{MaketasInzinieriusPagrindinis.png}
		\item Iš pateikto kreipinių sąrašo išsirinkti sau dominantį kreipinį \referToPicture{MaketasInzinieriusKreipiniuSarasas.png}
		\item Paspausti mygtuką "Perduoti Administratoriui" \referToPicture{MaketasInzinieriusKreipinioRodymas.png}
	\end{itemize}

\subsection{Vadovas}

	\subsubsection{Pasiekti administratoriaus, inžinieriaus veiksmus}
		
	\insertPicture[0.8]{maketas-vadovas-startas.png}{Vadovo pradinis vaizdas}
	
	Vadovui pasiekti administratoriaus bei inžinieriaus veiksmus galima pasinaudojus viršuje esančia juosta su mygtukais kiekvienam iš vaizdų.
	Pele paspaudus mygtuką "Administravimas" pereinama į Administratoriaus vaizdą (žr. Administratoriaus aprašymą).
	Paspaudus mygtuką "Kreipiniai" pereinama į Inžinieriaus vaizdą (žr. Inžinieriaus aprašymą).
	Pradinio Vadovo interfeiso maketas pateiktas \referToPicture{maketas-vadovas-startas.png}
	
	\subsubsection{Peržiūrėti statistiką}	
	
	\insertPicture[0.8]{maketas-vadovas-apzvalga}{Vadovo "Apžvalgos" vaizdas} 	
	
	\insertPicture[0.8]{maketas-statistika}{Statistikos vaizdas} 
	
	Vadovui pasiekti vėluojančių kreipinių statistiką galima iš "Apžvalgos" ekrano (Žr. \referToPicture{maketas-vadovas-apzvalga}), paspaudus statistikos mygtuką.
	Statistika pateikiama diagramomis bei sąrašais. Žr. \referToPicture{maketas-statistika}
	
	\textit{Pastaba: "Apžvalga" - vadovo pradinis ekranas - rezervuotas papildomam funkcionalumui, pagal kurį dokumentas bus papildytas.}
	
\subsection{Klientas}

	\subsubsection{Pateikti kreipinį}
	
	\insertPicture[0.8]{maketas-klientas-naujas_kreipinys.png}{Naujo kreipinio pateikimo maketas}
	
	Klientui, norint pateikti kreipinį, reikia paspausti mygtuką "Kreipiniai" ir tada mygtuką "Naujas kreipinys".
	Dešinėje lango pusėje atsiranda naujo kreipinio forma. Joje reikia užpildyti laukus:
	
	\begin{itemize}
		\item Įrašyti temą
		\item Aprašyti kreipinį
		\item Pasirinkti paslaugą
		\item Pasirinkti kreipinio tipą
	\end{itemize}
	
	Užpildžius naujo kreipinio formą reikia paspausti mygtuką "Siųsti".

	\subsubsection{Įvertinti atliktą kreipinį}
	
	\insertPicture[0.8]{maketas-klientas-ivertinti.png}{Atlikto kreipinio įvertinimas}
	
	Klientui, norint įvertinti atliktą kreipinį, reikia paspausti mygtuką "Kreipiniai" ir tada pasirinkti kreipinį.
	Dešinėje lango pusėje atsiranda laukas su išsiųstu kreipinio aprašymu, gautu atsakymu ir vertinimo forma.
	Vertinant atliktą kreipinį reikia perskaityti gautą atsakymą, vertinimo formoje pasirinkti įvertinimą ir paspausti mygtuką "Pateikti".
	
\subsection{Administratorius}

	\subsubsection{Registru tvarkymas}
	
	Administratoriui norint tvarkyti registrus reikia pereiti į registru langą. Pasirinkti reikiamą registrų tipą. 
	Priklausomai nuo reikiamo veiksmo:
	
	\begin{itemize}
		\item Šalinti irašą.
		\item Pridėti irašą.
		\item Redaguoti irašą.
		\itrm Duomenu importas.
	\end{itemize}
	
	\insertPicture[0.8]{Administratorius - tvarkyti registrus.png}{Registrų tvarkymo maketas}
	
	Jei veiksmas: šalinti irašą. Lange \referToPicture{Administratorius - tvarkyti registrus.png} reikia pasirinkti registro tipą ir ištrinti eilutę iš lentelės.

	Jei veiksmas: pridėti irašą. Lange \referToPicture{Administratorius - tvarkyti registrus.png} reikia pasirinkti registro tipą ir sukurti eilutę lentelėje.

	Jei veiksmas: redaguoti irašą. Lange \referToPicture{Administratorius - tvarkyti registrus.png} reikia pasirinkti registro tipą, pakeisti reikiamo registro eilutės duomenis.

	Jei veiksmas: duomenu importas. Lange \referToPicture{Administratorius - tvarkyti registrus.png} reikia spausti migtuka importuoti ir pasirinkti duomenų failą.