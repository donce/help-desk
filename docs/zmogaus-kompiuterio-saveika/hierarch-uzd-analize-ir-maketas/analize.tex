
\section{Analizė}

\subsection{Inžinieriaus užduočių analizė}

	\subsubsection{Kreipinio išsprendimas}
	
		\begin{itemize}
			\item \textbf{Įvestis}
				\subitem {Vykdomas Kreipinys}
			\item \textbf{Išvestis} 
				\subitem {Išspręstas kreipinys}
		\end{itemize}
		
	\paragraph{Užduoties dekompozicija}
	
	Užduoties dekompozicjai panaudojome Hierarchinės Užduočių Analizės metodus.
	Kreipinio išsprendimo užduotį skirstome hierarchiškai pagal abstraktumo lygmenis. 
	Žr. \referToPicture{Inzinierius-ispresti.png}
 		
	\insertPicture[0.8]{Inzinierius-ispresti.png}{Kreipinio išsprendimo analize}
	
	\subsubsection{Nusprendimas neišspręsti kreipinio}
	
		\begin{itemize}
			\item \textbf{Įvestis}
				\subitem {Vykdomas Kreipinys}
			\item \textbf{Išvestis} 
				\subitem {Atmestas kreipinys}
		\end{itemize}
		
	\paragraph{Užduoties dekompozicija} 
	
	Užduoties dekompozicjai panaudojome Hierarchinės Užduočių Analizės metodus.
	Kreipinio atmetimo užduotį skirstome hierarchiškai pagal abstraktumo lygmenis. 
	Žr. \referToPicture{Inzinierius-atmesti.png}
 		
	\insertPicture[0.8]{Inzinierius-atmesti.png}{Kreipinio atmetimo analizė}
	
	\subsubsection{Kreipinio gražinimas}	
	
		\begin{itemize}
			\item \textbf{Įvestis}
				\subitem {Vykdomas Kreipinys}
			\item \textbf{Išvestis} 
				\subitem {Kreipinys neturintis vykdančiojo Inžinieriaus}
				\subitem {Įspėtas administratorius apie kreipinio gražinimą}
		\end{itemize}

	\paragraph{Užduoties dekompozicija} 

	Užduoties dekompozicijai panaudojome Hierarchinės Užduočių Analizės metodus.
	Kreipinio atmetimo užduotį skirstome hierarchiškai pagal abstraktumo lygmenis. 
	Žr. \referToPicture{Inzinierius-grazinti.png}
 		
	\insertPicture[0.8]{Inzinierius-grazinti.png}{Kreipinio atmetimo analizė}

\subsection{Vadovo užduočių analizė}

	\subsubsection{Pasiekti inžinieriaus veiksmus}

	Užduotis neturi konkrečios įvesties. 
	Rezultatas - vadovas gali vykdyti inžinieriaus užduotis.
		
	\paragraph{Užduoties dekompozicija}
	
	Užduoties dekompozicijai panaudojome Hierarchinės Užduočių Analizės metodus.
	Kreipinio išsprendimo užduotį skirstome hierarchiškai pagal abstraktumo lygmenis. 
	Žr. \referToPicture{vadovas-admin.png}
 		
	\insertPicture[0.6]{vadovas-eng.png}{Inžinieriaus veiksmų pasiekimo analizė}

	\subsubsection{Pasiekti administratoriaus veiksmus}

	Užduotis neturi konkrečios įvesties. 
	Rezultatas - vadovas gali vykdyti administratoriaus užduotis.
		
	\paragraph{Užduoties dekompozicija}
	
	Užduoties dekompozicijai panaudojome Hierarchinės Užduočių Analizės metodus.
	Kreipinio išsprendimo užduotį skirstome hierarchiškai pagal abstraktumo lygmenis. 
	Žr. \referToPicture{vadovas-eng.png}	
 		
	\insertPicture[0.6]{vadovas-admin.png}{Administratoriaus veiksmų pasiekimo analizė}

	\subsubsection{Perskirti kreipinį kitam inžinieriui}

		\begin{itemize}
			\item \textbf{Įvestis}
				\subitem {Vykdomas Kreipinys}
				\subitem {Naujas vykdytojas}
			\item \textbf{Išvestis} 
				\subitem {Kreipinys su pakeistu jį vykdančiu inžinieriumi}
		\end{itemize}
		
	\paragraph{Užduoties dekompozicija}
	
	Užduoties dekompozicijai panaudojome Hierarchinės Užduočių Analizės metodus.
	Kreipinio išsprendimo užduotį skirstome hierarchiškai pagal abstraktumo lygmenis. 
	Žr. \referToPicture{vadovas-perskirti.png}
 	
	\insertPicture[0.6]{vadovas-perskirti.png}{Kreipinio išsprendimo analize}

\subsection{Administratoriaus užduočių analizė}

	\subsubsection{Kreipinio registravimas}

		\begin{itemize}
			\item \textbf{Įvestis}
				\subitem {Kliento prašymas užregistruoti kreipinį}
			\item \textbf{Išvestis} 
				\subitem {Užregistruotas kreipinys}
		\end{itemize}
		
	\subsubsection{Užduoties dekompozicija}

	Užduoties dekompozicjai panaudojome Hierarchinės Užduočių Analizės metodus.

	\subsubsection{Kreipinio paskyrimas inžinieriui}

		\begin{itemize}
			\item \textbf{Įvestis}
				\subitem {Inžinieriui nepriskirtas kreipinys}
			\item \textbf{Išvestis} 
				\subitem {Inžinieriui priskirtas kreipinys}
		\end{itemize}
		
	\subsubsection{Užduoties dekompozicija}

	Užduoties dekompozicjai panaudojome Hierarchinės Užduočių Analizės metodus.
	
	\subsubsection{Registrų tvarkymas}

		\begin{itemize}
			\item \textbf{Įvestis}
				\subitem {Registo pridėjimas/šalinimas/redagavimas}
			\item \textbf{Išvestis} 
				\subitem {Papildyta registru aibė}
		\end{itemize}
		
	\subsubsection{Užduoties dekompozicija}

	Užduoties dekompozicjai panaudojome Hierarchinės Užduočių Analizės metodus.
	Žr. \referToPicture{admin-tvarkyti_registrus.png}
	\insertPicture[0.6]{admin-tvarkyti_registrus.png}{Registru tvarkymo analize}
	\subsubsection{Duomenų importavimas iš struktūros}

		\begin{itemize}
			\item \textbf{Įvestis}
				\subitem {Duomenų struktūros failas}
			\item \textbf{Išvestis} 
				\subitem {Duomenys sistemoje}
		\end{itemize}
		
	\subsubsection{Užduoties dekompozicija}

	Užduoties dekompozicjai panaudojome Hierarchinės Užduočių Analizės metodus.
\subsection{Kliento užduočių analizė}

	\subsection{Pateikti kreipinį savitarnos svetainėje}
		
		\begin{itemize}
			\item \textbf{Įvestis}
				\subitem {Kilo incidentas/neaiškumai susiję su naudojamomis paslaugomis}
			\item \textbf{Išvestis} 
				\subitem {Pateiktas kreipinys}
		\end{itemize}

	\subsubsection{Užduoties dekompozicija}

	Užduoties dekompozicjai panaudojome Hierarchinės Užduočių Analizės metodus.
	Kreipinio pateikimo užduotį skirstome hierarchiškai pagal abstraktumo lygmenis. Žr. \referToPicture{Klientas-pateikti.png}

	
