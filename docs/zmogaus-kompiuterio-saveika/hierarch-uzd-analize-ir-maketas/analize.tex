\section{Analizė}

\subsection{Bendrų užduočių analizė}

	\subsubsection{Prisijungimas}
	
		\begin{itemize}
			\item \textbf{Įvestis}
				\subitem {Prisijungimo vardas}
				\subitem {Slaptažodis}
			\item \textbf{Išvestis} 
				\subitem {Vartotojas prisijungęs prie sistemos}
		\end{itemize}
		
		
	\paragraph{Užduoties dekompozicija}
	
	\insertPicture[0.8]{prisijungti}{Kreipinio išsprendimo analize}
	
	Užduoties dekompozicijai panaudojome Hierarchinės Užduočių Analizės metodus.
	Kreipinio išsprendimo užduotį skirstome hierarchiškai pagal abstraktumo lygmenis. 
	Žr. \referToPicture{prisijungti}

\subsection{Inžinieriaus užduočių analizė}
	
	\subsubsection{Kreipinio išsprendimas}
	
		\begin{itemize}
			\item \textbf{Įvestis}
				\subitem {Vykdomas Kreipinys}
			\item \textbf{Išvestis} 
				\subitem {Išspręstas kreipinys}
		\end{itemize}
		
		
	\paragraph{Užduoties dekompozicija}
	
	\insertPicture[0.8]{Inzinierius-ispresti}{Kreipinio išsprendimo analize}
	
	Užduoties dekompozicijai panaudojome Hierarchinės Užduočių Analizės metodus.
	Kreipinio išsprendimo užduotį skirstome hierarchiškai pagal abstraktumo lygmenis. 
	Žr. \referToPicture{Inzinierius-ispresti}
	
 		
	\subsubsection{Nusprendimas neišspręsti kreipinio}
	
		\begin{itemize}
			\item \textbf{Įvestis}
				\subitem {Vykdomas Kreipinys}
			\item \textbf{Išvestis} 
				\subitem {Atmestas kreipinys}
		\end{itemize}
		
		
	\paragraph{Užduoties dekompozicija} 
	
	\insertPicture[0.8]{Inzinierius-atmesti}{Kreipinio atmetimo analizė}
	
	Užduoties dekompozicjai panaudojome Hierarchinės Užduočių Analizės metodus.
	Kreipinio atmetimo užduotį skirstome hierarchiškai pagal abstraktumo lygmenis. 
	Žr. \referToPicture{Inzinierius-atmesti}
	
 		
	\subsubsection{Kreipinio gražinimas}	
	
		\begin{itemize}
			\item \textbf{Įvestis}
				\subitem {Vykdomas Kreipinys}
			\item \textbf{Išvestis} 
				\subitem {Kreipinys neturintis vykdančiojo Inžinieriaus}
				\subitem {Įspėtas administratorius apie kreipinio gražinimą}
		\end{itemize}


	\paragraph{Užduoties dekompozicija} 

	\insertPicture[0.8]{Inzinierius-grazinti}{Kreipinio atmetimo analizė}

	Užduoties dekompozicijai panaudojome Hierarchinės Užduočių Analizės metodus.
	Kreipinio atmetimo užduotį skirstome hierarchiškai pagal abstraktumo lygmenis. 
	Žr. \referToPicture{Inzinierius-grazinti}
	
 		
\subsection{Vadovo užduočių analizė}

	\subsubsection{Pasiekti inžinieriaus veiksmus}

	Užduotis neturi konkrečios įvesties. 
	Rezultatas - vadovas gali vykdyti inžinieriaus užduotis.
		
		
	\paragraph{Užduoties dekompozicija}
	
	\insertPicture[0.6]{vadovas-eng}{Inžinieriaus veiksmų pasiekimo analizė}

	Užduoties dekompozicijai panaudojome Hierarchinės Užduočių Analizės metodus.
	Kreipinio išsprendimo užduotį skirstome hierarchiškai pagal abstraktumo lygmenis. 
	Žr. \referToPicture{vadovas-eng}
 		
 		
	\subsubsection{Pasiekti administratoriaus veiksmus}

	Užduotis neturi konkrečios įvesties. 
	Rezultatas - vadovas gali vykdyti administratoriaus užduotis.
		
		
	\paragraph{Užduoties dekompozicija}
	
	\insertPicture[0.6]{vadovas-admin}{Administratoriaus veiksmų pasiekimo analizė}

	Užduoties dekompozicijai panaudojome Hierarchinės Užduočių Analizės metodus.
	Kreipinio išsprendimo užduotį skirstome hierarchiškai pagal abstraktumo lygmenis. 
	
	Žr. \referToPicture{vadovas-admin}	
	
 		
	\subsubsection{Perskirti kreipinį kitam inžinieriui}

		\begin{itemize}
			\item \textbf{Įvestis}
				\subitem {Vykdomas Kreipinys}
				\subitem {Naujas vykdytojas}
			\item \textbf{Išvestis} 
				\subitem {Kreipinys su pakeistu jį vykdančiu inžinieriumi}
		\end{itemize}
		
		
	\paragraph{Užduoties dekompozicija}
	
	\insertPicture[0.6]{vadovas-perskirti}{Kreipinio išsprendimo analize}

	Užduoties dekompozicijai panaudojome Hierarchinės Užduočių Analizės metodus.
	Kreipinio išsprendimo užduotį skirstome hierarchiškai pagal abstraktumo lygmenis. 
	
	Žr. \referToPicture{vadovas-perskirti}
	
	\subsubsection{Peržiūrėti statistiką}
	%% TODO: Statistikos analizė

		\begin{itemize}
			\item \textbf{Įvestis}
				\subitem {Sistemoje esantys duomenys}
			\item \textbf{Išvestis} 
				\subitem {Pateikti statistiniai duomenys}
		\end{itemize}
		
		
	\paragraph{Užduoties dekompozicija}
	
	\insertPicture[0.6]{maketas-statistika}{Statistikos peržiūros užduoties analizė}

	Užduoties dekompozicijai panaudojome Hierarchinės Užduočių Analizės metodus.
	Kreipinio išsprendimo užduotį skirstome hierarchiškai pagal abstraktumo lygmenis. 
	
	Žr. \referToPicture{maketas-statistika}
	
 	
\subsection{Administratoriaus užduočių analizė}

	\subsubsection{Kreipinio registravimas}

		\begin{itemize}
			\item \textbf{Įvestis}
				\subitem {Kliento prašymas užregistruoti kreipinį}
			\item \textbf{Išvestis} 
				\subitem {Užregistruotas kreipinys}
		\end{itemize}
		
		
	\paragraph{Užduoties dekompozicija}

	\insertPicture[0.6]{admin-registruoti.png}{Kreipinio registravimo analizė}

	Užduoties dekompozicijai panaudojome Hierarchinės Užduočių Analizės metodus.
	Žr. \referToPicture{admin-registruoti.png}
	
	
	\subsubsection{Kreipinio paskyrimas inžinieriui}

		\begin{itemize}
			\item \textbf{Įvestis}
				\subitem {Inžinieriui nepriskirtas kreipinys}
			\item \textbf{Išvestis} 
				\subitem {Inžinieriui priskirtas kreipinys}
		\end{itemize}
		
		
	\paragraph{Užduoties dekompozicija}


	\insertPicture[0.6]{admin-priskirti.png}{Kreipinio paskyrimo inžinieriui analizė}
	
	Užduoties dekompozicijai panaudojome Hierarchinės Užduočių Analizės metodus.
	Žr. \referToPicture{admin-priskirti}

	
	\subsubsection{Registrų tvarkymas}

		\begin{itemize}
			\item \textbf{Įvestis}
				\subitem {Registro pridėjimas/šalinimas/redagavimas}
			\item \textbf{Išvestis} 
				\subitem {pakeisti sistemos duomenis}
		\end{itemize}
		
		
	\paragraph{Užduoties dekompozicija}

	\insertPicture[0.6]{admin-tvarkyti-registrus}{Registrų tvarkymo analizė}
	
	Užduoties dekompozicijai panaudojome Hierarchinės Užduočių Analizės metodus.
	Žr. \referToPicture{admin-tvarkyti-registrus}
	
	
	\subsubsection{Duomenų importavimas iš struktūros}

		\begin{itemize}
			\item` \textbf{Įvestis}
				\subitem {Duomenų struktūros failas}
			\item \textbf{Išvestis} 
				\subitem {Duomenys sistemoje}
		\end{itemize}

		
	\paragraph{Užduoties dekompozicija}

	\insertPicture[0.6]{admin-importuoti}{Duomenų importavimo analizė}
	
	Užduoties dekompozicijai panaudojome Hierarchinės Užduočių Analizės metodus.
	Žr. \referToPicture{admin-importuoti}
	
	
\subsection{Kliento užduočių analizė}

	\subsubsection{Pateikti kreipinį savitarnos svetainėje}
		
		\begin{itemize}
			\item \textbf{Įvestis}
				\subitem {Kilo incidentas/neaiškumai susiję su naudojamomis paslaugomis}
			\item \textbf{Išvestis} 
				\subitem {Pateiktas kreipinys}
		\end{itemize}


	\paragraph{Užduoties dekompozicija}

	\insertPicture{klientas-naujas_kreipinys}{Kreipinio pateikimo savitarnos sistemoje analizė}

	Užduoties dekompozicijai panaudojome Hierarchinės Užduočių Analizės metodus.
	Kreipinio pateikimo užduotį skirstome hierarchiškai pagal abstraktumo lygmenis. 
	Žr. \referToPicture{klientas-naujas_kreipinys}
	
	\subsubsection{Įvertinti atliktą kreipinį}
		
		\begin{itemize}
			\item \textbf{Įvestis}
				\subitem {Atliktas kreipinys}
			\item \textbf{Išvestis} 
				\subitem {Pateiktas įvertinimas}
		\end{itemize}


	\paragraph{Užduoties dekompozicija}

	\insertPicture{klientas-ivertinimas}{Kreipinio pateikimo savitarnos sistemoje analizė}

	Užduoties dekompozicijai panaudojome Hierarchinės Užduočių Analizės metodus.
	Kreipinio pateikimo užduotį skirstome hierarchiškai pagal abstraktumo lygmenis. 
	Žr. \referToPicture{klientas-ivertinimas}

