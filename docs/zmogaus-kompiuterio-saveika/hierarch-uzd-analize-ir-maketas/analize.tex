\section{Analizė}

\subsection {Inžinieriaus užduočių analizė}
	\subsection {Kreipinio išsprendimas}
		\begin{itemize}
			\item \textbf{Įvestis}
				\subitem {Vykdomas Kreipinys}
			\item \textbf{Išvestis} 
				\subitem {Išspręstas kreipinys}
		\end{itemize}
	\subsubsection {Užduoties dekompozicija}
	Užduoties dekompozicjai panaudojome Hierarchinės Užduočių Analizės metodus.
 		
	\insertPicture[0.8]{images/Inzinierius-ispresti}{Kreipinio išsprendimo analize}
	Kreipinio išsprendimo užduotį skirstome hierarchiškai pagal abstraktumo lygmenis. Žr. \referToPicture{images/Inzinierius-ispresti}

	\subsection {Nusprendimas neišspręsti kreipinio}
		\begin{itemize}
			\item \textbf{Įvestis}
				\subitem {Vykdomas Kreipinys}
			\item \textbf{Išvestis} 
				\subitem {Atmestas kreipinys}
		\end{itemize}
	\subsubsection {Užduoties dekompozicija}
	Užduoties dekompozicjai panaudojome Hierarchinės Užduočių Analizės metodus.
 		
	\insertPicture[0.8]{images/Inzinierius-atmesti}{Kreipinio atmetimo analizė}
	Kreipinio atmetimo užduotį skirstome hierarchiškai pagal abstraktumo lygmenis. Žr. \referToPicture{images/Inzinierius-atmesti}
	\subsection {Kreipinio gražinimas}	
		\begin{itemize}
			\item \textbf{Įvestis}
				\subitem {Vykdomas Kreipinys}
			\item \textbf{Išvestis} 
				\subitem {Kreipinys neturintis vykdančiojo Inžinieriaus}
				\subitem {Įspėtas administratorius apie kreipinio gražinimą}
		\end{itemize}
	\subsubsection {Užduoties dekompozicija}
	Užduoties dekompozicjai panaudojome Hierarchinės Užduočių Analizės metodus.
 		
	\insertPicture[0.8]{images/Inzinierius-grazinti}{Kreipinio atmetimo analizė}
	Kreipinio atmetimo užduotį skirstome hierarchiškai pagal abstraktumo lygmenis. Žr. \referToPicture{images/Inzinierius-grazinti}
