
\section{Analizė}

\subsection {Inžinieriaus užduočių analizė}

	\subsection {Kreipinio išsprendimas}
	
		\begin{itemize}
			\item \textbf{Įvestis}
				\subitem {Vykdomas Kreipinys}
			\item \textbf{Išvestis} 
				\subitem {Išspręstas kreipinys}
		\end{itemize}
		
	\subsubsection {Užduoties dekompozicija}
	
	Užduoties dekompozicjai panaudojome Hierarchinės Užduočių Analizės metodus.
 		
	\insertPicture[0.8]{Inzinierius-ispresti.png}{Kreipinio išsprendimo analize}
	
	Kreipinio išsprendimo užduotį skirstome hierarchiškai pagal abstraktumo lygmenis. Žr. \referToPicture{Inzinierius-ispresti.png}

	\subsection {Nusprendimas neišspręsti kreipinio}
	
		\begin{itemize}
			\item \textbf{Įvestis}
				\subitem {Vykdomas Kreipinys}
			\item \textbf{Išvestis} 
				\subitem {Atmestas kreipinys}
		\end{itemize}
		
	\subsubsection {Užduoties dekompozicija}
	
	Užduoties dekompozicjai panaudojome Hierarchinės Užduočių Analizės metodus.
 		
	\insertPicture[0.8]{Inzinierius-atmesti.png}{Kreipinio atmetimo analizė}
	
	Kreipinio atmetimo užduotį skirstome hierarchiškai pagal abstraktumo lygmenis. Žr. \referToPicture{Inzinierius-atmesti.png}
	
	\subsection {Kreipinio gražinimas}	
	
		\begin{itemize}
			\item \textbf{Įvestis}
				\subitem {Vykdomas Kreipinys}
			\item \textbf{Išvestis} 
				\subitem {Kreipinys neturintis vykdančiojo Inžinieriaus}
				\subitem {Įspėtas administratorius apie kreipinio gražinimą}
		\end{itemize}

	\subsubsection {Užduoties dekompozicija}

	Užduoties dekompozicjai panaudojome Hierarchinės Užduočių Analizės metodus.
 		
	\insertPicture[0.8]{Inzinierius-grazinti.png}{Kreipinio atmetimo analizė}

	Kreipinio atmetimo užduotį skirstome hierarchiškai pagal abstraktumo lygmenis. Žr. \referToPicture{Inzinierius-grazinti.png}
