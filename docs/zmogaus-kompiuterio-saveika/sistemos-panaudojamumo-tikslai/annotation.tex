%%
%% annotation.tex
%% Dokumento anotacija
%%
%% Patogesniam darbu ir redagavimui - komandas, 
%% palengvinančias darbą su anotacija, palieku šiame dokumente


%	--- Anotacija su INDĖLIU ---
%	Naudojimas:
%	\anotacijaIndelis{AUTORIUS}{EL@PAŠTAS.com}{INDĖLIS}
\newcommand{\anotacijaIndelis}[3]{
	\textbf{#1}
	\begin{flushleft}
	\hspace*{1cm}
	Kontaktai: #2
	\\
	\hspace*{1cm}
	Indėlis: #3
	\end{flushleft}
}

%	--- Anotacija ---
%	Naudojimas:
%	\anotacija{AUTORIUS}{EL@PAŠTAS.com}
\newcommand{\anotacija}[2]{
	\textbf{#1}
	\begin{flushleft}
	\hspace*{1cm}
	Kontaktai: #2
	\end{flushleft}
}

\section*{Anotacija}

		Darbo tikslas – išnagrinėti naudotojų ir užduočių charakteristikas, specifikuoti iš pastarųjų išplaukiančius esminių užduočių panaudojamumo tikslus.
		Atsižvelgiant į keliamus tikslus rasti įkvėpiančias dizaino idėjas. \\
		
		\textbf{Bibliografinis aprašas:}\\
		Kristina Moroz-Lapin Žmogaus ir kompiuterio sąveika. Vilniaus universitetas, 2008. 248 p. ISBN 978-9955-680-99-4
		\\
		
		\textbf{Darbą atliko:}\\
		%% Autoriai:
		
		\anotacijaIndelis{Karolis Jocevičius PS1}{karolis.jocevicius@gmail.com}{Vadovo personos aprašymas}
		
		\anotacijaIndelis{Laima Čižiūtė PS1}{ugne.ciziute@gmail.com.com}{Kliento personos aprašymas}
		
		\anotacijaIndelis{Rytis Karpuška PS1}{jauleris@gmail.com.com}{Inžinieriaus personos aprašymas}

		\anotacijaIndelis{Donatas Kučinskas PS1}{donce.lt@gmail.com.com}{Administratoriaus personos aprašymas}

		\anotacijaIndelis{Oleg Koldun PS1}{okoldun@gmail.com}{Kategorijos, aprašymų struktūra}