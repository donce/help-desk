\section{Suinteresuotų asmenų kategorijos}

	\subsection{Pirminiai asmenys}
	
	\textbf{Administratorius} ir \textbf{Inžinierius} yra pirminiai sistemos vartotojai, nuolat dirbantys su sistema.
	
	\subsection{Antriniai asmenys}
	
	\textbf{Vadovas} ir \textbf{Klientas} yra antriniai vartotojai, dirbantys su sistema tiesiogiai, bet gana retai.

\section{Personos}

	\subsection{Vadovas}
		
		\subsubsection{Siekiai projekte}
		
		Laiku pastebėti įmonės darbe iškilusias problemas, prastai dirbančius darbuotojus.
		
		\subsubsection{Charakteristika}
		
		\begin{itemize}
			\item \textbf{Naudojamos IT priemonės:}
				\subitem Naršyklės
				\subitem Biuro programų paketai
			\item \textbf{Darbo aplinka.} Tikslių duomenų nėra. 
			Tai galėtų būti biuro patalpa su stacionariu ar nešiojamu kompiuteriu.
			Pastaruoju atveju - su galimybe dirbti nuotoliniu būdu.
		\end{itemize}
			
		\subsubsection{Tipas}
		
		Vadovas yra vidutiniškai patyręs nedažnas vartotojas.
		
		\subsubsection{Kompiuterizuojamos užduotys}
		
		\begin{itemize}
			\item Įmonės darbo stebėjimas
			\item Visos inžinieriaus ir administratoriaus užduotys (žr. reikalavimų specifikacijos funkcinius reikalavimus F9, F10)
			\item Perskirti kreipinius iš vieno inžinieriaus kitam (žr. reikalavimų specifikacijos funkcinius reikalavimus F9, F11)
		\end{itemize}
		
		\subsubsection{Problemos esamoje situacijoje}
		
		Sudėtinga stebėti įmonėje vykstančius procesus, iškylančias problemas, perskirstyti darbus kitiems darbuotojams.
		
		\subsubsection{Patobulintos sąveikos vizija}
		
		Sistemoje vadovui turi būti suteikta galimybė realiu laiku stebėti įmonėje vykstantį darbą, matyti kilusias dėmesio reikalaujančias situacijas.
		
		\subsubsection{Būsimos sistemos kompiuterizuotų užduočių panaudojimo tikslai}
		
			\setcounter{tocdepth}{5} \setcounter{secnumdepth}{5}
			
			\paragraph{Riboto naudojimo etapas}
			
			Šiame etape vartotojas mokysis naudotis sistema.
			
			Prisijungimo ekranas paprastas ir atpažįstamas iš daugelio kitų web sistemų.
			Jį perprasti turėtų užtekti 1-2 minučių.

			Vadovo vaizdas, kuriame matoma visa įmonėje vykstanti veikla.
			Pirmą kartą susipažinti su jame pateikiama informacija užtruktų 10-15 minučių.
			Užduoties perskyrimas kitam inžinieriui užtruktų ne daugiau 5 minučių.
						
			Inžinieriaus ir administratoriaus funkcijos, pasiekiamos vadovo, atitinka šių personų kompiuterizuotų užduočių panaudojimo tikslų aprašymus.
			
			\setcounter{tocdepth}{5} \setcounter{secnumdepth}{5}
			
			\paragraph{Pilno naudojimo etapas}
			
			Šiame etape vadovas pilnai naudoja sistemos galimybes ir yra su jomis susipažinęs.
			Prisijungimas užtrunka ne daugiau 30 sekundžių. Informacijos peržvelgimas vadovo ekrane užtrunka ne daugiau 2 minučių.
			
			
	
	\subsection{Administratorius}
	
		\subsubsection{Siekiai projekte}
		
		Laiku fiksuoti ir skirstyti kreipinius. 
		
		\subsubsection{Charakteristika}
		
		\begin{itemize}
			\item \textbf{Naudojamos IT priemonės:}
				\subitem Naršyklės
				\subitem El. paštas
			\item \textbf{Darbo aplinka.}
			Tai biuro patalpa su stacionariu ar nešiojamu kompiuteriu, telefonu.
		\end{itemize}
			
		\subsubsection{Tipas}
		
		Administratorius yra patyręs naudotojas. Sistemos sudėtingumas nedaro įtakos darbui.
		
		\subsubsection{Kompiuterizuojamos užduotys}
		
		\begin{itemize}
			\item Kreipinio registravimas – kiekvieną kartą kleintui pateikus kreipinį telefonu arba el. paštu.
			\item Kreipinių skirstymas - kiekvieną kartą atsiradus naujam kreipiniui arba inžinieriui atsisakius vykdyti kreipinį.
		\end{itemize}
		
		\subsubsection{Problemos esamoje situacijoje}
		
		Sunku efektyviai paskirstyti kreipinius.
		
		\subsubsection{Patobulintos sąveikos vizija}
		
		Inžinierių sąrašas turėtų būti surikiuotas pagal jų užimtumo lygį.
		
		\subsubsection{Būsimos sistemos kompiuterizuotų užduočių panaudojimo tikslai}
		
			\setcounter{tocdepth}{5} \setcounter{secnumdepth}{5}
			
			\paragraph{Riboto naudojimo etapas}
			
			Šiame etape vartotojas mokysis naudotis sistema.
			
			Prisijungimo ekranas paprastas ir atpažįstamas iš daugelio kitų web sistemų.
			Jį perprasti turėtų užtekti 1-2 minučių.

			Pirmą kartą susipažinti su ekrane pateikiama informacija užtruktų 30-40 minučių.
			Administratorius turi gebėti pilnai suprasti sistemos veikimo principus po apmokymų.
			
			\setcounter{tocdepth}{5} \setcounter{secnumdepth}{5}
			
			\paragraph{Pilno naudojimo etapas}
			
			Šiame etape aministratorius pilnai naudoja sistemos galimybes ir yra su jomis susipažinęs.
			Prisijungimas užtrunka ne daugiau 30 sekundžių. Kreipinio registravimas užtrunka ne daugiau 2 minučių.
			Kreipinių paskirstymas užtrunka ne daugiau 5 minučių.
			
	\subsection{Inžinierius}
	
	\subsection{Klientas}
			
		\subsubsection{Siekiai projekte}
		
		Palengvinti kreipinių fiksavimą.
		
		\subsubsection{Charakteristika}
		
		\begin{itemize}
			\item \textbf{Naudojamos IT priemonės:}
				\subitem Naršyklės
				\subitem El. paštas
			\item \textbf{Darbo aplinka.} Tikslių duomenų nėra. 
			Tai galėtų būti biuro patalpa su stacionariu ar nešiojamu kompiuteriu, telefonu.
			Pastaruoju atveju - su galimybe dirbti nuotoliniu būdu.
		\end{itemize}
			
		\subsubsection{Tipas}
		
		Klientas yra naujokas. Jam reikalingas kuo paprastesnis interfeisas, leidžiantis suprasti, kaip naudotis sistema.
		
		\subsubsection{Kompiuterizuojamos užduotys}
		
		\begin{itemize}
			\item Kreipinio registravimas – kiekvieną kartą atsiradus nesklandumų su paslauga arba kilus klausimui
		\end{itemize}
		
		\subsubsection{Problemos esamoje situacijoje}
		
		Informacija gali būti suteikta per vėlai, incidentų sprendimas gali vėluoti, tai gali kainuoti daug įmonės lėšų.
		
		\subsubsection{Patobulintos sąveikos vizija}
		
		Pateikti laukiamą informciją laiku, vengti incidentų sprendimo vėlavimo.
		
		\subsubsection{Būsimos sistemos kompiuterizuotų užduočių panaudojimo tikslai}
		
			\setcounter{tocdepth}{5} \setcounter{secnumdepth}{5}
			
			\paragraph{Riboto naudojimo etapas}
			
			Šiame etape vartotojas mokysis naudotis sistema.
			
			Prisijungimo ekranas paprastas ir atpažįstamas iš daugelio kitų web sistemų.
			Jį perprasti turėtų užtekti 1-2 minučių.

			Pirmą kartą susipažinti su ekrane pateikiama informacija užtruktų 10-15 minučių.
			Klientas turi be aplinkinių pagalbos suprasti sistemos veikimo principus.
			
			\setcounter{tocdepth}{5} \setcounter{secnumdepth}{5}
			
			\paragraph{Pilno naudojimo etapas}
			
			Šiame etape klientas pilnai naudoja sistemos galimybes ir yra su jomis susipažinęs.
			Prisijungimas užtrunka ne daugiau 30 sekundžių. Kreipinio registravimas užtrunka ne daugiau 5 minučių.
			Kreipinio būsenos tikrinimas užtrunka ne daugiau 2 minučių.
			Sistema turi būti atspari visiems klaidingiems naudotojo veiksmams, turi būti galimybė atšaukti visus veiksmus.
			
\section{Įkvėpiančios idėjos}

"Teambox" svetainė puikiai realizuoja komandinio darbo užduočių skirstymą ir nepateikia perteklinės informacijos, todėl vartotojo sąsajos idėja galima būtų pritaikyti šiame projekte.

"JIRA" sistema turi daug galimybių filtruoti užduotis bei dirbti su daug projektų. Tai būtų galima pritaikyti klientų skirstymui.

"Kanban view" yra vaizdavimo būdas, kuri galima pritaikyti kreipinių vaizdavimui.

"Travian" žaidimas puikus bendros informacijos atvaizdavimo pavyzdys, jis pateikia duomenis apie dabartinį progresą realiu laiku. Tai būtų galima pritaikyti vadovo stebėjimo moduliui.

"Facebook" socialinis tinklapis turi puikų ispėjimų atvaizdavimo būdą, kurį galima pritaikyti vartotojų perspėjimui apie naujus įvykius įmonėje.

	