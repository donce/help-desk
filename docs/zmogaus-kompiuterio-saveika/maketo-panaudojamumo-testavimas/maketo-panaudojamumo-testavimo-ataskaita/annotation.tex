%%
%% annotation.tex
%% Dokumento anotacija
%%
%% Patogesniam darbu ir redagavimui - komandas, 
%% palengvinančias darbą su anotacija, palieku šiame dokumente


%	--- Anotacija su INDĖLIU ---
%	Naudojimas:
%	\anotacijaIndelis{AUTORIUS}{EL@PAŠTAS.com}{INDĖLIS}
\newcommand{\anotacijaIndelis}[3]{
	\textbf{#1}
	\begin{flushleft}
	\hspace*{1cm}
	Kontaktai: #2
	\\
	\hspace*{1cm}
	Indėlis: #3
	\end{flushleft}
}

%	--- Anotacija ---
%	Naudojimas:
%	\anotacija{AUTORIUS}{EL@PAŠTAS.com}
\newcommand{\anotacija}[2]{
	\textbf{#1}
	\begin{flushleft}
	\hspace*{1cm}
	Kontaktai: #2
	\end{flushleft}
}

\section*{Anotacija}

		Šiame darbe patobulintą interfeisą testuoja asmenys, atitinkantys antrame laboratoriniame darbe pateiktą personos apibūdinimą. \\
		
		\textbf{Bibliografinis aprašas:}\\
		Kristina Moroz-Lapin Žmogaus ir kompiuterio sąveika. Vilniaus universitetas, 2008. 248 p. ISBN 978-9955-680-99-4
		\\
		
		\textbf{Darbą atliko:}\\
		%% Autoriai:

		% TODO: Pakeisti indėlį
		\anotacija{Karolis Jocevičius PS1}{karolis.jocevicius@gmail.com}
		
		\anotacija{Laima Čižiūtė PS1}{ugne.ciziute@gmail.com}
		
		\anotacija{Rytis Karpuška PS1}{jauleris@gmail.com}

		\anotacija{Donatas Kučinskas PS1}{donce.lt@gmail.com}

		\anotacija{Oleg Koldun PS1}{okoldun@gmail.com}
