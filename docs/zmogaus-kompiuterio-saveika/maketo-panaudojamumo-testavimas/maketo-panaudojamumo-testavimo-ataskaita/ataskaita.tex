\section{Santrauka}

	Vertinimą vykdė FluffySoft komandos nariai.
	5 dalyviai buvo pasirinkti atsitiktinai iš artimiausios aplinkos, kuriuos stebėjo komandos nariai ir fiksavo jų sąveiką su kompiuteriu.
	Dėl pasirinktos aplinkos stebimi dalyviai buvo įvairaus amžiaus ir įvairių sričių atstovai, visi turėjo patirties naudojantis kompiuteriu ankščiau.
	Vertinimo aplinka – patalpa su kompiuteriu ir be pašalinių trikdžių.
	Siekiant atskleisti galimas panaudojamumo problemas buvo pasirinktas „įvertinimo atitikimo taisyklėms“ metodas.
	
	Atlikus išsamią dalyvių atsiliepimų bei rezultatų analizę buvo nustatyti šie teigiami aspektai:
	\begin{itemize}
		\item Visi testo dalyviai labai gerai atsiliepė apie sistemos dizainą.
		\item Testo dalyviams labai patiko, kad tame pačiame lange klientas gali matyti ir kreipinių sąrašą, ir pasirinktą kreipinį
	\end{itemize}
	
	Pagrindinės rastos problemos sistemoje, į kurias reiktų atsižvelgti kuriant galutinę sistemos versiją:
	\begin{itemize}
		\item Labiausiai išryškėjo problemos kilusios dėl angliško meniu.
		Šis defektas turėtų būti sprendžiamas verčiant pavadinimus į lietuvių kalbą.
		\item Vykdant testavimą dalis sistemos funkcionalumo buvo išvis nerasta.
		Taip nutiko todėl, kad sistema dar nėra pilnai užbaigta.
		\item Kai kurios interaktyvios sritys atrodė kaip paprastos nuorodos.
		Pažeista darna turėtų būti taisoma nuorodas pakeičiant mygtukais.
		\item Sistemos grįžtamasis ryšys taip pat turėjo defektų.
		Pranešimų apie sėkmingai atliktą operaciją išvis nėra.
		Šis defektas galėtų būti ištaisytas pranešimais apie sėkmingai įvykdytą užuotį.
		\item Taip pat nebuvo rasti klaidų pranešimai, kurie turėtų pranešti apie įvykusias klaidas.
		Šis defektas galėtų būti ištaisytas pranešimais apie nesėkmingai įvykdytą užduotį ir padarytas klaidas.
	\end{itemize}

\section{Testavimo aprašas}

	\subsection{Testuojamos užduotys}
	
		\begin{enumerate}
			\item Prisijungimas
				\subitem Sėkmės kriterijus: prisijungta prie sistemos
				\subitem Matas: užtrunka neilgiau nei minutę
			\item Kreipinio išsprendimas (Inžinierius)
				\subitem Sėkmės kriterijus: kreipinys pažymėtas išspręstu
				\subitem Matas: sprendimo kelias rąstas intuityviai
			\item Nusprendimas neišspręsti kreipinio (Inžinierius)
				\subitem Sėkmės kriterijus: kreipinys sėkmingai atmestas
				\subitem Matas: sprendimo kelias rąstas intuityviai
			\item Kreipinio grąžinimas (Inžinierius)
				\subitem Sėkmės kriterijus: kreipinys sėkmingai grąžintas administratoriui
				\subitem Matas: sprendimo kelias rąstas intuityviai
			\item Peržiūrėti statistiką (Vadovas)
				\subitem Sėkmės kriterijus: Sėkmingai atvertas statistikos langas
				\subitem Matas: sprendimo kelias rąstas intuityviai
			\item Kreipinio registravimas (Administratorius)
				\subitem Sėkmės kriterijus: Sėkmingai užregistruotas kreipinys
				\subitem Matas: užtrunka neilgiau nei 5 minutes
			\item Kreipinio paskyrimas inžinieriui (Administratorius)
				\subitem Sėkmės kriterijus: kreipinys inžinieriui paskirtas sėkmingai
				\subitem Matas: užtrunka neilgiau nei 2 minutes
			\item Duomenų importavimas iš struktūros (Administratorius)
				\subitem Sėkmės kriterijus: struktūra sėkmingai importuota
				\subitem Matas: užtrunka neilgiau nei 2 minutes
			\item Pateikti kreipinį savitarnos svetainėje (Klientas)
				\subitem Sėkmės kriterijus: sėkmingai pateiktas kreipinys
				\subitem Matas: užtrunka neilgiau nei 5 minutes
			\item Įvertinti atliktą kreipinį (Klientas)
				\subitem Sėkmės kriterijus: kreipinys sėkmingai įvertintas
				\subitem Matas: sprendimo kelias rąstas intuityviai
		\end{enumerate}
	
	\subsection{Metodas}
	
		Siekiant atskleisti galimas panaudojamumo problemas buvo pasirinktas „įvertinimo atitikimo taisyklėms“ metodas.
		Testo vykdymo procedūra buvo pasirinkta tokia:
		\begin{enumerate}
			\item Prie stalo su kompiuteriu buvo pasodintas testo dalyvis. Kompiuterio naršyklėje buvo atidarytas sistemos prisijungimo puslapis.
			\item Testo organizatorius papasakoja apie sistemos paskirtį.
			\item Testuotojui pateikiama sutikimo testuoti anketa bei klausimynas apie jo patirtį.
			\item Užpildžius klausimyną testuotojui yra pateikiamos užduotys, kurias jis turi įvykdyti ir įvertinti.
			\item Po visų užduočių atlikimo padėkojama testo dalyviams už dalyvavimą.
		\end{enumerate}
	
	\subsection{Aplinka}
	
		Nešiojami kompiuteriai (15.6' ekranas) su Windows bei Ubuntu operacinėmis sistemomis.
		Testavimo vykdymo vieta - neformali.
	
	\subsection{Dalyviai}
	
		\begin{itemize}
			\item V.P. pareigos - studentas. Šiose pareigose 0-1 metų. Nėra grupės vadovas. Informacinių technologijų patirtis 3 ir daugiau metų.
			Dalyvis naudoja teksto procesorius, skaičiuokles, duomenų bazes, pristatymų rengimo priemones.
			Per dieną namuose praleidžia 3-8 valandas prie kompiuterio, o darbe 1-3 valandas.
			Kompiuterį naudoja darbui bei laisvalaikiui.
			\item M.P. pareigos - direktoriaus pavaduotojas. Šiose pareigose daugiau nei 3 metus. Nėra grupės vadovas. Informacinių technologijų patirtis 3 ir daugiau metų.
			Dalyvis naudoja teksto procesorius, skaičiuokles, duomenų bazes, pristatymų rengimo priemones.
			Per dieną namuose praleidžia 1-3 valandas prie kompiuterio, o darbe 3-8 valandas.
			Kompiuterį naudoja darbui bei laisvalaikiui.
			\item A.Č. pareigos - administratorė. Šiose pareigose daugiau nei 3 metus. Nėra grupės vadovė. Informacinių technologijų patirtis 3 ir daugiau metų.
			Dalyvė naudoja teksto procesorius, skaičiuokles, duomenų bazes.
			Namuose prie kompiuterio nesėdi, o darbe per diena praleidžia 3-8 valandas.
			Kompiuterį naudoja darbui bei laisvalaikiui.
			\item L.J. pareigos - prekybos ir logistikos vadovas. Šiose pareigose daugiau nei 3 metus. Yra 10 žmonių grupės vadovas. Informacinių technologijų patirtis 3 ir daugiau metų.
			Dalyvis naudoja teksto procesorius, skaičiuokles, duomenų bazes, pristatymų rengimo priemones.
			Per dieną namuose praleidžia 1-3 valandas prie kompiuterio, o darbe 3-8 valandas.
			Kompiuterį naudoja darbui bei laisvalaikiui.
			\item N.J. pareigos - vyr. buhalterė. Šiose pareigose daugiau nei 3 metus. Yra 5 žmonių grupės vadovė. Informacinių technologijų patirtis 3 ir daugiau metų.
			Dalyvis naudoja teksto procesorius, skaičiuokles, duomenų bazes.
			Per dieną namuose praleidžia 1-3 valandas prie kompiuterio, o darbe 3-8 valandas.
			Kompiuterį naudoja darbui bei laisvalaikiui.
		\end{itemize}
	

\section{Testavimo rezultatai}

	\subsection{Užduočių vykdymo rezultatai}
	
		Lentelė pildoma stebint dalyvio sąveiką su kompiuteriu kiekvienos užduoties metu.
		Dalyvio užduoties atlikimo greitis - vertinimas minutėmis.
		Dalyvio intuityvi orientacija sistemoje užduoties metu - vertinimas nuo 0 (visiškai nesigaudo sistemoje) iki 5 (puikiai gaudosi sistemoje).
		Suklydimų ir grįžimų atgal skaičius - klaidų ir grįžimų skaičius prieš atliekant teisingai užduotį.
		Lentelėje pateikti visų dalyvių užduočių vertinimo vidurkiai.
		Kiekvieno atskirai dalyvio užduočių vertinimai pridėti priede Nr.1. Dalyvių rezultatų lentelės.
	
	\begin{table}[ht] 
		\caption{Dalyvių užduočių vertinimo vidurkiai} % title of Table 
		\centering % used for centering table 
		\begin{tabular}{c c c c} % centered columns (2 columns) 
			\hline\hline %inserts double horizontal lines 
			Užduotis & Atlikimo greitis & Intuityvi orientacija  & Suklydimų ir grįžimų atgal sk.\\ [0.5pt] % inserts table 
			%heading 
			\hline % inserts single horizontal line 
			1 & 0,7 & 5 & 0\\
			2 & 0,58 & 4,6 & 0,2\\
			3 & 0,66 & 4,2 & 0,6\\
			4 & 0,38 & 5 & 0\\
			5 & 0,58 & 4,6 & 0\\
			6 & 4,5 & 4,2 & 0,2\\
			7 & 0,94 & 4,6 & 0\\
			8 & 2,4 & 3 & 1,4\\
			9 & 2,7 & 5 & 0\\
			10 & 0,54 & 4,8 & 0\\
			\hline %inserts single line 
		\end{tabular} 
		\label{table:vertintojai} % is used to refer this table in the text 
	\end{table} 
	
	\subsection{Dalyvių komentarai}
	
		Dalyvių komentarai:
		
		\begin{enumerate}
			\item Prisijungimas
			\item Kreipinio išsprendimas (Inžinierius)
				\subitem A.Č. norėtų, kad kreipinio išsprendimo mygtukas būtų apačioje.
			\item Nusprendimas neišspręsti kreipinio (Inžinierius)
				\subitem A.Č. norėtų, kad kreipinio atmetimo mygtukas būtų apačioje.
				\subitem M.P ir L.J. nesuprato, kuo skiriasi kreipinio atmetimo mygtukas nuo kreipinio grąžinimo administratoriui mygtuko.
			\item Kreipinio grąžinimas (Inžinierius)
				\subitem A.Č. norėtų, kad kreipinio grąžinimo administratoriui mygtukas būtų apačioje.
			\item Peržiūrėti statistiką (Vadovas)
				\subitem L.J. norėtų daugiau statistikos.
			\item Kreipinio registravimas (Administratorius)
				\subitem J.L. nesuprato, kuo skiriasi INC nuo REQ.
			\item Kreipinio paskyrimas inžinieriui (Administratorius)
			\item Duomenų importavimas iš struktūros (Administratorius)
				\subitem Visiems dalyviams nebuvo aišku, kurioje kortelėje ieškoti šio funkcionalumo.
			\item Pateikti kreipinį savitarnos svetainėje (Klientas)
				\subitem J.L. nesuprato, kuo skiriasi INC nuo REQ.
			\item Įvertinti atliktą kreipinį (Klientas)
		\end{enumerate}

\section{Rekomendacijos}

	Defektai išvardyti prioritetine tvarka
	\begin{itemize}
		\item Atlikus tyrimą, labiausiai išryškėjo problemos kilusios dėl angliško meniu.
		Testo dalyviai sugaišdavo sąlyginai daug laiko ieškodami reikiamų meniu pasirinkimų užduotims įvykdyti.
		Tai trukdė vykdyti užduotis, tačiau labiau patyrusiam vartotojui tai nekeltų problemų.
		Šis defektas turėtų būti sprendžiamas verčiant pavadinimus į lietuvių kalbą.
		\item Vykdant testavimą dalis sistemos funkcionalumo buvo išvis nerasta.
		Taip nutiko todėl, kad sistema dar nėra pilnai užbaigta.
		\item Kai kurios interaktyvios sritys atrodė kaip paprastos nuorodos, o vartotojas pripratęs prie mygtukų tiesiog nepastebėdavo šių nuorodų.
		Pažeista darna turėtų būti taisoma nuorodas pakeičiant mygtukais.
		\item Sistemos grįžtamasis ryšys taip pat turėjo defektų.
		Pranešimų apie sėkmingai atliktą operaciją išvis nėra.
		Testo dalyviui buvo sunku suprasti, kada jis teisingai atlikdavo užduotį.
		Šis defektas galėtų būti ištaisytas pranešimais apie sėkmingai įvykdytą užuotį.
		\item Taip pat nebuvo rasti klaidų pranešimai, kurie turėtų pranešti apie įvykusias klaidas.
		Tokie defektai klaidino vartotoją, tačiau didelės įtakos darbui su sistema nedarė.
		Vartotojas prarado galimybę pastebėti tikrai kilusias problemas.
		Šis defektas galėtų būti ištaisytas pranešimais apie nesėkmingai įvykdytą užduotį ir padarytas klaidas.
	\end{itemize}

\section{Priedai}

	\begin{enumerate}
		\item Dalyvių rezultatų lentelės
	\end{enumerate}
